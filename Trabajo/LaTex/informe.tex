\documentclass[11pt,compsoc]{IEEEtran}
\usepackage[spanish]{babel}
\usepackage{amsmath,amsfonts}
\usepackage{algorithmic}
\usepackage{array}
\usepackage[caption=false,font=normalsize,labelfont=sf,textfont=sf]{subfig}
\usepackage{textcomp}
\usepackage{stfloats}
\usepackage{url}
\usepackage{verbatim}
\usepackage{graphicx}
\hyphenation{op-tical net-works semi-conduc-tor IEEE-Xplore}
\def\BibTeX{{\rm B\kern-.05em{\sc i\kern-.025em b}\kern-.08em
		T\kern-.1667em\lower.7ex\hbox{E}\kern-.125emX}}
\usepackage{balance}
\begin{document}
	\title{El procesador CELL, desde un enfoque
		histórico}
	\author{Ramiro Barcala Roca, Valentin Angrigiani, Gabriel Hackl}
		
	\markboth{Organizacion del Computador, Catedra Marchi, 2024}%
	{How to Use the IEEEtran \LaTeX \ Templates}
	\maketitle
	
	\begin{abstract}
		Este trabajo tratará el procesador CELL de Sony. Se toma un enfoque investigativo y contrastante entre arquitecturas del CELL, y otras de la época (y la actualidad). Las aplicaciones principales para las que fue diseñado, y las aplicaciones que se descubrieron luego junto  con su importancia histórica. Conceptos básicos de computación heterogénea (distintos nucleos). Analisis a futuro relacionandolo a todo.
	\end{abstract}
	
	\begin{IEEEkeywords}
		SPE,PPE,SIMD,.
	\end{IEEEkeywords}
	
	\section{Introducción}
	\IEEEPARstart{A} mediados de los 2000, la empresa Sony empieza a investigar y desarrollar su sistema de entretenimiento "Play-Station 3". Todo esto en un mercado competitivo frente a otras marcas como Microsoft y Nintendo. Terminan con un procesador multi-core, el cual tiene como particularidades los sub-nucleos. Sony tambien quería estandarizar su procesador en dispositivos de todas las gamas.
	
	Sin embargo, tanto el producto como su procesador,fueron un fracaso comercial. Sin embargo, se descubrió un uso investigativo/científico.
	
	
	\section{Arquitecturas de la epoca}
	\noindent 
	
	\subsection{Competencia de Sony}
	\noindent 
	
	\subsection{Arquitecturas pasadas de Sony}
	\noindent 
	
	\subsection{Arquitecturas varias}
	\noindent 
	
	
	
	
	
	\section{Arquitectura CELL}	% MODULOS,ARQUITECTURAS INTERNAS, CONVENCIONES, INSTRUCCIONES, etc
	\noindent 
	
	\subsection{Modulos, en general y separado} 
	\noindent 
	
	\subsection{Interconexion} 
	\noindent 
	
	\subsection{Computacion heterogenea }
	\noindent 
	
	\subsection{Computacion heterogenea en el Cell }% SIMD
	\noindent 
	
	
	
	
	\section{Inconveniencias}
	\noindent Debido a que programar los distintos SSP requerían mucho tiempo, una gran cantidad  de desarrolladores pasaron por alto los programas apartes que requerian los mismos, y se centraban unicamente en el PPE. De modo que, todas las tareas eran realizadas por el PPE, lo cual provocó bajos rendimientos en los programas que corria el CELL, ya que sus SPE's quedaban inutilizados.
	
	Muchas tareas que se podian paralelizar, no se estaban ejecutando como debían, ni por los sub-nucleos especializados para eso. 
	
	Esto 
	
	
	
	
	
	
	
	\section{Usos del CELL}
	\noindent
	
	\subsection{Puntos fuerte}
	\noindent
	
	\subsection{Research and Development}
	\noindent
	
	
	
	\section{Remanentes del Cell en la Actualidad}
	\noindent 
	
	
	
	
	
	
	
	\section{Analisis a futuro}
	\noindent 
	
	
	
	
	
	
	\section{Conclusion}
	\noindent
	
	
		\begin{thebibliography}{1}
			
			\bibitem{ams}
			{\it{Laboratorio de Arquitecturas Avanzadas con Cell y PlayStation 3}}, Universitat de València, por Fernando Pardo y Jose A. Boluda.
			
			\bibitem{oxford}
			{\it{Un vistazo al pasado, ¿cómo de potente fue PS3? }},https://www.3djuegos.com/ps3/noticias/un-vistazo-al-pasado-como-de-potente-fue-ps3-190331-91210
		
			\bibitem{ams}
			{\it{The PlayStation Supercomputer}}, https://www.datacenterdynamics.com/en/analysis/the-playstation-supercomputer/
			
			\bibitem{ams}
			{\it{The Untold Story of the Cell Processor: Sony's Pioneering Technology in Gaming}}, https://www.gameversedaily.com/post/the-untold-story-of-the-cell-processor-sony-s-pioneering-technology-in-gaming
			
			\bibitem{ams}
			{\it{Console wars: A rare bright spot in the gloomy technology industry, video games are growing up}}, 
			https://www.economist.com/business/2002/06/20/console-wars
			
			\bibitem{ams}
			{\it{Console wars}}, 			https://www.ft.com/content/ef24f36e-5c54-11dc-9cc9-0000779fd2ac
			
			\bibitem{ams}
			{\it{Console wars}}, 			
			https://www.gainesville.com/story/news/2006/11/11/waging-console-war/31502430007/
			
			\bibitem{ams}
			{\it{PlayStation 3, Console Wars and the Costs of Complexity}}, 			
			https://techliberation.com/2006/09/07/playstation-3-console-wars-the-costs-of-complexity/
			
			\bibitem{ams}
			{\it{Console giants square up}}, 			
			http://news.bbc.co.uk/2/hi/entertainment/2002112.stm
			
			\bibitem{ams}
			{\it{Report from the video game wars: Wii vs. PS3 vs. Xbox360}}, 
			https://fabricegrinda.com/report-from-the-video-game-wars-wii-vs-ps3-vs-xbox360/
		
		\end{thebibliography}
		
			
	
		
\end{document}
