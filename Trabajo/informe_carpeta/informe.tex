\documentclass[12pt,compsoc]{IEEEtran}
\usepackage[spanish]{babel}
\usepackage{amsmath,amsfonts}
\usepackage{algorithmic}
\usepackage{array}
\usepackage[caption=false,font=normalsize,labelfont=sf,textfont=sf]{subfig}
\usepackage{textcomp}
\usepackage{stfloats}
\usepackage{url}
\usepackage{verbatim}
\usepackage{graphicx}
\hyphenation{op-tical net-works semi-conduc-tor IEEE-Xplore}
\def\BibTeX{{\rm B\kern-.05em{\sc i\kern-.025em b}\kern-.08em
		T\kern-.1667em\lower.7ex\hbox{E}\kern-.125emX}}
\usepackage{balance}
\begin{document}
	\title{Procesador CELL de Sony, un enfoque histórico}
	\author{Ramiro Barcala Roca, Valentin Angrigiani, Gabriel Hackl}
		
	\markboth{Organizacion del Computador, Catedra Marchi, 2024}%
	{How to Use the IEEEtran \LaTeX \ Templates}
	\maketitle
	
	\begin{abstract}
		Este trabajo tratará el procesador CELL de Sony. Se toma un enfoque investigativo y contrastante entre arquitecturas del CELL, y otras de la época (y la actualidad). Las aplicaciones principales para las que fue diseñado, y las aplicaciones que se descubrieron luego junto  con su importancia histórica. Conceptos básicos de computación heterogénea (distintos nucleos). Analisis a futuro relacionandolo a todo.
	\end{abstract}
	
	\begin{IEEEkeywords}
		Sub-nucleos, .
	\end{IEEEkeywords}
	
	\section{Introduction}
	\IEEEPARstart{A} mediados de los 2000, la empresa Sony empieza a investigar y desarrollar su sistema de entretenimiento "Play-Station 3". Todo esto en un mercado competitivo frente a otras marcas como Microsoft y Nintendo. Terminan con un procesador multi-core, el cual tiene como particularidades los sub-nucleos. Sony tambien quería estandarizar su procesador en dispositivos de todas las gamas.
	
	Sin embargo, tanto el producto como su procesador,fueron un fracaso comercial. Sin embargo, se descubrió un uso investigativo/científico.
	
	\section{Introduction}
	\IEEEPARstart{A} mediados de los 2000, la empresa Sony empieza a investigar y desarrollar su sistema de entretenimiento "Play-Station 3". Todo esto en un mercado competitivo frente a otras marcas como Microsoft y Nintendo. Terminan con un procesador multi-core, el cual tiene como particularidades los sub-nucleos. Sony tambien quería estandarizar su procesador en dispositivos de todas las gamas.
	
	Sin embargo, tanto el producto como su procesador,fueron un fracaso comercial. Sin embargo, se descubrió un uso investigativo/científico.
	
	
		\begin{thebibliography}{1}
			
			\bibitem{ams}
			{\it{Mathematics into Type}}, American Mathematical Society. Online available: 
			
			\bibitem{oxford}
			T.W. Chaundy, P.R. Barrett and C. Batey, {\it{The Printing of Mathematics}}, Oxford University Press. London, 1954.
			
			\bibitem{lacomp}{\it{The \LaTeX Companion}}, by F. Mittelbach and M. Goossens
			
			\bibitem{mmt}{\it{More Math into LaTeX}}, by G. Gr\"atzer
			
			\bibitem{amstyle}{\it{AMS-StyleGuide-online.pdf,}} published by the American Mathematical Society
			
			\bibitem{Sira3}
			H. Sira-Ramirez. ``On the sliding mode control of nonlinear systems,'' \textit{Systems \& Control Letters}, vol. 19, pp. 303--312, 1992.
			
			\bibitem{Levant}
			A. Levant. ``Exact differentiation of signals with unbounded higher derivatives,''  in \textit{Proceedings of the 45th IEEE Conference on Decision and Control}, San Diego, California, USA, pp. 5585--5590, 2006.
			
			\bibitem{Cedric}
			M. Fliess, C. Join, and H. Sira-Ramirez. ``Non-linear estimation is easy,'' \textit{International Journal of Modelling, Identification and Control}, vol. 4, no. 1, pp. 12--27, 2008.
			
			\bibitem{Ortega}
			R. Ortega, A. Astolfi, G. Bastin, and H. Rodriguez. ``Stabilization of food-chain systems using a port-controlled Hamiltonian description,'' in \textit{Proceedings of the American Control Conference}, Chicago, Illinois, USA, pp. 2245--2249, 2000.
			
		\end{thebibliography}
		
		\begin{IEEEbiographynophoto}{Jane Doe}
			Biography text here without a photo.
		\end{IEEEbiographynophoto}
		
		\begin{IEEEbiography}
			In this paragraph you can place your educational, professional background and research and other interests.\end{IEEEbiography}
		
		
\end{document}
